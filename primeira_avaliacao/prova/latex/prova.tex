%%%%%%%%%%%%%%%%%%%%%%%%%%%%%%%%%%%%%%%%%%%%%%%%%%%%%%%%%%%%%%%%%%%%%%%%%%%%%%%%%%%%
% Do not alter this block (unless you're familiar with LaTeX
\documentclass{article}
\usepackage[margin=1in]{geometry} 
\usepackage{amsmath,amsthm,amssymb,amsfonts, fancyhdr, color, comment, graphicx, environ}
\usepackage{xcolor}
\usepackage{mdframed}
\usepackage[shortlabels]{enumitem}
\usepackage{indentfirst}
\usepackage{hyperref}
\hypersetup{
    colorlinks=true,
    linkcolor=blue,
    filecolor=magenta,      
    urlcolor=blue,
}


\pagestyle{fancy}


\newenvironment{problem}[2][Questão]
    { \begin{mdframed}[backgroundcolor=gray!20] \textbf{#1 #2} \\}
    {  \end{mdframed}}

% Define solution environment
\newenvironment{solution}
    {\textit{Solução:}}
    {}

\renewcommand{\qed}{\quad\qedsymbol}

% prevent line break in inline mode
\binoppenalty=\maxdimen
\relpenalty=\maxdimen

%%%%%%%%%%%%%%%%%%%%%%%%%%%%%%%%%%%%%%%%%%%%%
%Fill in the appropriate information below
\lhead{Lucas de Nóvoa Martins Pinto - Primeira avaliação processos estocásticos PPGEE 2021.4}

%%%%%%%%%%%%%%%%%%%%%%%%%%%%%%%%%%%%%%%%%%%%%

\begin{document}

\begin{mdframed}[backgroundcolor=blue!20]

ALUNO: \textbf{Lucas de Nóvoa Martins Pinto} $\newline$
MATRICULA: \textbf{202100470044} 	

\textbf{OBS-1:}  Entregar resolução com comando digitado ou digitalizado. As resoluções podem ser digitadas ou podem ser fotos (nítidas) das questões resolvidas. Entregar documento 
único em pdf. 

\textbf{OBS-2:} Nesta prova somente serão consideradas respostas com resoluções detalhadas e  com  respostas  finais  corretas.  Resoluções  pela  metade  ou  com respostas  finais erradas serão desconsideradas. Desta forma, façam e refaçam a prova com \textbf{atenção!!!!!}
\end{mdframed}

\begin{problem}{1}
\textbf{[1 pt]} Questao\_01 Uma  caixa  contém  2000  componentes  dos  quais  5\%  são  defeituosos.  Uma segunda  caixa  contém  500  componentes  dos  quais  40\%  são  defeituosos.  Duas  outras caixas contém 1000 componentes cada, com 10\% de componentes defeituosos em cada 
uma destas caixas. É selecionada aleatoriamente uma das caixas acima e removido dela um único componente. Considerando que o componente retirado é examinado e constata-se que ele é defeituoso. Qual a probabilidade que ele tenha sido retirado da segunda caixa.

\end{problem}

\begin{solution}

Usando o teorema de bayes:

A - Evento escolhe a Urna dois

B - Evento escolher uma peça defeituosa no meio das peças normais

$P(A|B) = \frac{P(B/A) P(A)}{P(B)}$

$P(A) = \frac{500}{4500} = \frac{1}{9} $

$P(B) = \frac{200}{500} = \frac{2}{5} $

$P(B/A) = \frac{500}{4500} \times \frac{200}{500} = \frac{2}{45} $

Assim,

$P(A|B) = \frac{P(B/A) P(A)}{P(B)} = \frac{\frac{2}{45} \times \frac{2}{5}}{\frac{1}{9}}$

$P(A|B) = \frac{2}{45} \times \frac{2}{5} \times 9 = \frac{2}{5}$

\textbf{Resposta: $\frac{2}{5}$}


\end{solution}

\begin{problem}{2}
\textbf{[3 pts]} Questao\_2 Responda: 

\textbf{[1 pt]} a) Uma pessoa lança dois dados, um atrás do outro. Determine a probabilidade da soma dos dados lançados ser 7, visto que o primeiro dado lançado possui um número maior que o segundo dado lançado.

\textbf{[1 pt]} b) Sabendo-se que uma variável aleatória X assume os valores -1, 0, 1 e se o valor médio E[X]= 
0  e  o  segundo  momento,  E[X2]=  0,5. Determine  a  função  densidade  de  probabilidade  para  a 
variável aleatória discreta X. 
 
\textbf{[1 pt]} c) Seja R uma variável aleatória continua com função densidade de probabilidade

$f(R) = \frac{R}{16b^{2}} e^{-\frac{R^{2}}{32b^{2}}}$, QUANDO, $R \geq 0$

$f(R) = 0$, QUANDO, $R < 0$

Determine a função distribuição de probabilidade.  

\textbf{OBS para aluno:} deixe a expressão da distribuição de probabilidade em função de R.

\end{problem}

\begin{solution}

Considerando um dado com 6 lados

Dado, um par ordenado como sendo (a, b)

Se é um dado atrás do outro então o espaço amostral TOTAL SERIA:

\[ (1,1) (1,2) (1,3) (1,4) (1,5) (1,6) = 6 \newline \]
\[ (2,1) (2,2) (2,3) (2,4) (2,5) (2,6) = 6 \newline \]
\[ (3,1) (3,2) (3,3) (3,4) (3,5) (3,6) = 6 \newline \]
\[ (4,1) (4,2) (4,3) (4,4) (4,5) (4,6) = 6\newline \]
\[ (5,1) (5,2) (5,3) (5,4) (5,5) (5,6) = 6\newline \]
\[ (6,1) (6,2) (6,3) (6,4) (6,5) (6,6) = 6\newline \]

porém, como temos uma condição que o primeiro dado tem que ser maior que o segundo ou seja $a>b$


\[ (2,1) = 1 \newline \]
\[ (3,1) (3,2) = 2 \newline \]
\[ (4,1) (4,2) (4,3) = 3\newline \]
\[ (5,1) (5,2) (5,3) (5,4) = 4\newline \]
\[ (6,1) (6,2) (6,3) (6,4) (6,5) = 5\newline \]

entao, $1 + 2 + 3 + 4 + 5 = \textbf{15 Resultados possíveis}$

Caso soma dá 7 acontece quando:

$S = {(4,3) (5,2) (6,1)}\newline$ 

Assim, 

P(W) = $\frac{3}{15} = \frac{1}{5}$


\textbf{Resposta: $\frac{1}{5}$}

\hrulefill

b) Sabendo que E[X] = 0

Temos,

$\sum_{-1}^{1} p_{i} x_{i} = -1 \times pa + 0 \times pb + 1 \times pc$

$\sum_{-1}^{1} p_{i} x_{i} = -pa + pc = 0$

Para $E[x^{2}] = 0.5 = (-1)^{2} \times pa + 0^{2} \times pb + 1^{2} \times pc$

$E[x^{2}] = pa + pc = 0.5$

Sabendo também que, $pa+pb+pc = 1$

Como o pc aparece em todas as equações e fazendo com a segundo equacao $pa=pc$

temos, $pc + pc = 0.5$

$pc = 0,25$

O que implica dizer que, $pb = 0,5 \newline$


\textbf{Resposta:}


$f(x) = 0,25$, QUANDO, $x=-1$

$f(x) = 0,5$, QUANDO, $x=0$

$f(x) = 0,25$, QUANDO, $x=1$



\hrulefill

c) Toda função densidade de probabilidade tem que ser igual a 1: $\int_{-\infty}^{\infty} f(x) \, dx = 1$

O que implica disser que: $\int_{-\infty}^{\infty} (\frac{R}{16b^{2}} e^{\frac{-R^{2}}{32a}}) \, dx= 1$

$u = \frac{-R^{2}}{32a^{2}}$

Derivando temos: 

$du = -\frac{R}{16a^{2}} dr$

que é igual a 

$-du = \frac{R}{16a^{2}} dr$

$\int_{0}^{\infty} (\frac{R}{16b^{2}} e^{\frac{-R^{2}}{32a}}) \, dx = - \int_{0}^{\infty} e^{u} \, du$

$- \int_{0}^{\infty} e^{u} \, du = -e^{\frac{-R^{2}}{32u^{2}}}$

$-e^{-8} - e^{0} = 2$

A Função distribuição de probabilidade é dada como se segue:

$F(R) = \int f(R) dR = e^{-\frac{R^{2}}{32u^{2}}}$

Assim a distribuição é definida

$F(R) = 0$, QUANDO, $x<0$

$F(R) = e^{-\frac{R^{2}}{32u^{2}}}$, QUANDO, $0<x<\infty$

$F(R) = 1$, QUANDO, $x=\infty$

\end{solution}

\begin{problem}{3}
\textbf{[3 pts]} Questao\_3 Uma caixa contém 5 bolas pretas(p), 3 azuis(a) e 7 vermelhas(v). A experiencia aleatória 
consiste na realização de duas extrações sucessivas de uma bola sem reposição. Suponha que foi atribuída a seguinte pontuação: bola preta-1ponto; bola azul-2 pontos, bola vermelha-3 pontos. 
Considere a variável aleatória X, “soma dos pontos obtidos”. Determine:

\textbf{[1 pt]} a) $P(3 \leq x \leq 5)$ 

\textbf{[1 pt]} b) $P(X>3/X<6)$ 

\textbf{[1 pt]}  c) a função distribuição de X

\end{problem}

\begin{solution}

X -> A v.a. que é a soma dos pontos obitidos

5 Pretas (P) - 1 ponto/bola

3 Azuis (A)- 2 pontos/bola

7 Vermelhas (V) - 3 pontos/bola

E dado que (a,b) é o par ordenado que representa a primeira puxado e depois a segunda SEM REPOSIÇÃO

a)  $P(3 \leq x \leq 5) = P(3) + P(4) + P(5)$

$P(3) = (P,A), (A,P)$

A probabilidade de tirar preta depois azul é (P,A):

$\frac{5}{15} \times \frac{3}{14}$

A probabilidade de tirar azul depois preta é (A,P):

$\frac{3}{15} \times \frac{5}{14}$

$P(3) = \frac{5}{15} \times \frac{3}{14} + \frac{3}{15} \times \frac{5}{14}$

$P(3) = \frac{1}{7}$

Usando esse mesmo raciocinio vamos fazer para P(4) e P(5)

$P(4) = \frac{3}{15} \times \frac{2}{14} + \frac{5}{15} \times \frac{7}{14} + \frac{7}{15} \times \frac{5}{14}$

$P(4) = \frac{38}{105}$

$P(5) = \frac{3}{15} \times \frac{7}{14} + \frac{7}{15} \times \frac{3}{14}$

$P(5) = \frac{1}{5}$

Assim,

$P(3 \leq x \leq 5) = \frac{1}{7} + \frac{38}{105} + \frac{1}{5}$

$P(3 \leq x \leq 5) = \frac{74}{105}$

\hrulefill

\textbf{b) }

$P(X>3/X<6)$ 

$P(X>3/X<6) = \frac{P(x > 3 \cap x < 6)}{P(x<6)}$ 

$P(x > 3 \cap x < 6) = P(4) + P(5) = \frac{38}{105} + \frac{1}{5}$

$P(x > 3 \cap x < 6) = \frac{59}{105}$


$P(x < 6) = P(2) + P(3) + P(4) + P(5)$

$P(2) = \frac{5}{15} \times \frac{4}{14} = \frac{2}{21}$

$P(x < 6) = \frac{2}{21} + \frac{1}{7} + \frac{38}{105} + \frac{1}{5} = \frac{4}{5}$

Assim,

\textbf{$P(X>3/X<6) = \frac{\frac{59}{105}}{\frac{4}{5}} = \frac{59}{84}$}

\hrulefill

\textbf{c) } 

A função de distribuição de X

\begin{center}
\begin{tabular}{ |c|c|c|c|c|c|}
\hline

f(x) & $\frac{2}{21}$ & $\frac{2}{14}$ & $\frac{1}{7}$ & $\frac{38}{105}$ & $\frac{1}{5}$ \\
X & 2 & 3 & 4 & 5 & 6 \\
 
 \hline
\end{tabular}
\end{center}


A distribuição fica da seguinte forma:

$F(x) = 0$, QUANDO, $x<2$

$F(x) = \frac{2}{21}$, QUANDO, $2<x<3$

$F(x) = \frac{2}{21} + \frac{1}{7} = \frac{5}{21}$, QUANDO, $3<x<4$

$F(x) = \frac{5}{21} + \frac{38}{105} = \frac{3}{5}$, QUANDO, $4<x<5$

$F(x) = \frac{3}{5} + \frac{2}{10} = \frac{4}{5}$, QUANDO, $5<x<6$

$F(x) = 1$, QUANDO, $x>6$


\end{solution}

\begin{problem}{4}
\textbf{[1 pt]} Questao\_4 Seja X uma variável aleatória com função distribuição de probabilidade dada por:

$F(x) = 0$, QUANDO, $x < 0$

$F(x) = 3x^{2} - 2x^{3}$, QUANDO, $ 0 \leq x < 1$

$F(x) = 1$, QUANDO, $x \geq 0$

Determine $P(X \leq \frac{1}{2} | \frac{1}{3} < X < \frac{2}{3})$

\end{problem}

\begin{solution}

Observe que, $P(X \leq \frac{1}{2} | \frac{1}{3} < X < \frac{2}{3}) = \frac{P(X \leq \frac{1}{2} \cap \frac{1}{3} < X < \frac{2}{3})}  {P(\frac{1}{3} < X < \frac{2}{3})}$

COMO,

$\frac{1}{2} = 0.5$

$\frac{1}{3} = 0.333$

$\frac{2}{3} = 0.66$


Assim, se X é menor ou igual a meio já implica que ele seja maior que um terço e menos que dois terços: $X < \frac{1}{2} => \frac{1}{3} < X < \frac{2}{3}$

Com isso,

$P(X \leq \frac{1}{2} | \frac{1}{3} < X < \frac{2}{3}) = \frac{P(\frac{1}{3} < X < \frac{2}{3})}{P(\frac{1}{3} < X < \frac{2}{3})}$

Dessa forma, calculando a distribuição em $\frac{1}{3}<X<\frac{1}{2}$

$P(\frac{1}{3}<X<\frac{1}{2}) = \int_{1/3}^{1/2} f(x) \, dx = 3x^{2} - 2x^{3}|_{1/3}^{1/2}$

$P(\frac{1}{3}<X<\frac{1}{2}) = (\frac{3}{4} - \frac{2}{8}) - (3 \times \frac{1}{9} - 2 \times \frac{1}{27} = \frac{13}{54}$

No intervalo: $\frac{1}{3}<X<\frac{2}{3}$

Temos:

$P(\frac{1}{3}<X<\frac{2}{2}) = \int_{\frac{1}{3}}^{\frac{2}{3}} f(x) \, dx = 3x^{2} - 2x^{3}|_{1/3}^{2/3}$

$P(\frac{1}{3}<X<\frac{2}{2}) = (3 \times \frac{4}{9} - 2 \times \frac{8}{27}) - (3 \times \frac{1}{9} - 2 \times \frac{1}{27}) = \frac{26}{54}$

Agora podemos prosseguir:



$P(X \leq \frac{1}{2} | \frac{1}{3} < X < \frac{2}{3}) = \frac{\frac{13}{54}}{\frac{26}{54}} = \frac{13}{26} = \frac{1}{2}$

\textbf{Respostas 50\%}


\end{solution}


\end{document}





