%%%%%%%%%%%%%%%%%%%%%%%%%%%%%%%%%%%%%%%%%%%%%%%%%%%%%%%%%%%%%%%%%%%%%%%%%%%%%%%%%%%%
% Do not alter this block (unless you're familiar with LaTeX
\documentclass{article}
\usepackage[margin=1in]{geometry} 
\usepackage{amsmath,amsthm,amssymb,amsfonts, fancyhdr, color, comment, graphicx, environ}
\usepackage{xcolor}
\usepackage{mdframed}
\usepackage[shortlabels]{enumitem}
\usepackage{indentfirst}
\usepackage{hyperref}
\hypersetup{
    colorlinks=true,
    linkcolor=blue,
    filecolor=magenta,      
    urlcolor=blue,
}


\pagestyle{fancy}


\newenvironment{problem}[2][Questão]
    { \begin{mdframed}[backgroundcolor=gray!20] \textbf{#1 #2} \\}
    {  \end{mdframed}}

% Define solution environment
\newenvironment{solution}
    {\textit{Solução:}}
    {}

\renewcommand{\qed}{\quad\qedsymbol}

% prevent line break in inline mode
\binoppenalty=\maxdimen
\relpenalty=\maxdimen

%%%%%%%%%%%%%%%%%%%%%%%%%%%%%%%%%%%%%%%%%%%%%
%Fill in the appropriate information below
\lhead{Lucas de Nóvoa Martins Pinto - Segunda lista processos estocásticos PPGEE 2021.4}

%%%%%%%%%%%%%%%%%%%%%%%%%%%%%%%%%%%%%%%%%%%%%

\begin{document}

\begin{mdframed}[backgroundcolor=blue!20]
segunda lista da primeira avaliação de Processos Estocasticos turma 2021.4
\end{mdframed}

\begin{problem}{1}
\textbf{[0.04pts]} Questao\_01 Uma variável aleatória discreta X tem função densidade de probabilidade dada por:
\begin{center}
\begin{tabular}{ |c|c|c|c|c| }
\hline
 x1        & 0    & 1   & 2 & 3   \\
 P(X = x1) & 1/10 & 1/5 & m & 1/10 \\
 \hline
\end{tabular}
\end{center}


 a)   Determine o valor de m. 
 
 b) Construa a função distribuição de probabilidade da variável aleatória X.

\end{problem}

\begin{solution}

\textbf{a)}

$\sum_{-\infty}^{\infty} p(x_{i}) = 1$

$\frac{1}{10} + \frac{1}{5} + m + \frac{1}{10} = 1$

$m = \frac{3}{5}$

\hrulefill

\textbf{b)}

$F_{x}(x) =  \sum_{k}^{} p_{x}(x_{k}) u(x - x_{k})$

$u(x) = 0$ quando $x < 0 $

$u(x) = 1$ quando $x \geq 0$

\end{solution}

\begin{problem}{2}
\textbf{[0.04pts]} Questao\_2 Seja X a variável aleatória que representa o número de computadores vendidos, por dia, numa loja de um Centro Comercial. A função densiadde de probabilidade da v.a. X dada por 

\begin{center}
\begin{tabular}{ |c|c|c|c|c|}
\hline

x & 0 & 1 & 2 & 3 \\
P (X = x) & 1/6 & p & 1/30 & 3/10 \\

 \hline
\end{tabular}
\end{center}

a) Determine a probabilidade de vender um computador por dia.

b) Calcule P (1 ≤ X < 4); P (X > 2); P (X ≤ 1); P (X > 1/ X < 3).

\end{problem}

\begin{solution}

$X ->$ número de computadores vendidos por dia

\textbf{a)}

$\sum_{0}^{3} p(x_{i}) = 1$

$\frac{1}{6} + p + \frac{1}{30} + \frac{3}{10} = 1$

$p = \frac{15}{30}$ OU $p=\frac{1}{2}$

\hrulefill

\textbf{b)} P (1 ≤ X < 4) = $ P(1) + P(2) + P(3)$
$\frac{1}{2} + \frac{1}{30} + \frac{3}{10} = \frac{25}{30} = \frac{5}{6}$

$P (X > 2)$ = $P(3) = \frac{3}{10}$

$P (X \leq 1)$ = $P(0) + P(1) = \frac{1}{6} + \frac{1}{2} = \frac{4}{6}$

$P (X > 1/ X < 3)$ = $\frac{P(1<X<3)}{P(X<3)} = \frac{P(2)}{P(0)+P(1)+P(2)} = \frac{\frac{1}{30}}{\frac{1}{6}+\frac{1}{2}+\frac{1}{30}} = \frac{1}{21}$



\end{solution}

\begin{problem}{3}
\textbf{[0.04pts]} Questao\_3 Seja X um v.a. que nos indica o número de automoveis procurados por um dia num certo stand de vendas. A função de probabilidade da v.a.  X     dada por:

\begin{center}
\begin{tabular}{ |c|c|c|c|c|c|}
\hline

x &0 & 1 & 2 & 3 & 4 \\ 
P(X=x) & 1/20 & p & q & 1/3 & 1/4 \\

 \hline
\end{tabular}
\end{center}

a) Sabendo que, em 75\% dos dias são procurados pelo menos dois automóveis, calcule p e q. 

b) Calcule a probabilidade de virem a ser procurados 3 automóveis, num dia em que as procuras foram em número de pelo menos  dois.

\end{problem}

\begin{solution}

$P(X \geq 2) = \frac{75}{100} = \frac{3}{4}$ \\
$P(X \geq 2) = q + \frac{1}{3} + \frac{1}{4} = \frac{3}{4}$ \\
$q = \frac{3}{4} - \frac{1}{3} - \frac{1}{4} = \frac{2}{12} = \frac{1}{6}$ \\
$q=\frac{1}{6}$ 

$\sum_{0}^{4} P(x_{i}) = 1$ \\
$\frac{1}{20} + p + \frac{1}{6} + \frac{1}{3} + \frac{1}{4} = 1$ \\
$p = 1 - \frac{1}{20} - \frac{9}{12}$ \\
$p = \frac{12}{60} = \frac{1}{5}$

\hrulefill

b)
$P(\frac{x=3}{x \geq 2}) = \frac{P(x=3)}{P(x \geq 2)} = \frac{\frac{1}{3}}{\frac{1}{6}+\frac{1}{3}+\frac{1}{4}} = \frac{\frac{1}{3}}{\frac{9}{12}} = \frac{4}{9}$

\end{solution}

\begin{problem}{4}
\textbf{[0.04pts]} Questao\_4 Seja X uma variável aleatória discreta com a seguinte função densidade de probabilidade: 


\begin{center}
\begin{tabular}{ |c|c|c|c|c|c|}
\hline

x & −2 & −1 & 0 & 1 & 2 \\
P (X = x) & 0,1 & 0,3 & 0,1 & 0,2 & 0, 3 \\

 \hline
\end{tabular}
\end{center}

a)Calcule  E[X]  e  Var[X]

b)Desenhe a função de distribuição de probabilidade de X.  

c)Desenhe a distribuição de probabilidade da variável aleatória Y  = $X^{2}$.

\end{problem}

\begin{solution}

a) E(x) e Var(x)

$E(x) = \sum_{i=1}^{n} x_{i} p_{x}(x_{i})$ \\
$E(x) = u_{x} = -2 0.1 - 1 0.3 + 1 0.2 + 2 0.3$
$u_{x} = 0.3$

$Var(x) = \sigma_{x}^{2} = \sum_{-\infty}^{\infty} (x_{i}u_{x})^{2} p(x_{i})$ \\

$ Var(x) = (-2-0.3)^{2} 0.1 + (-1-0.3)^2 0.3 + (0 - 0.3)^{2}0.1 + (1-0.3)^{2}0.2+ (2-0.3)^{2}0.3$ \\

$Var(x) = 0.529 + 0.507 + 0.098 + 0.009 + 0.867$ \\
$Var(x) = 2.01$

\end{solution}

\begin{problem}{5}
\textbf{[0.04pts]} Questao\_5 A variável aleatória X tem a seguinte distribuição de probabilidades:


\begin{center}
\begin{tabular}{ |c|c|c|c|c|c|}
\hline

x & 2 & 3 & 5 & 8 \\
Probabilidade & 0,2 & 0,4 & 0,3 & 0,1 \\

 \hline
\end{tabular}
\end{center}

Determine o seguinte: 
 
a) $P(X \leq 3)$ 
 
b) $P(X > 2,5)$

c) $P(2,7 < X < 5,1) $

d) $E(X)$ 

e) $V(X)$ (variância ) 

\end{problem}

\begin{solution}

a) $P (X \leq 3) = P(2) + P(3) = 0.2 + 0.4 = 0.6$

b) $P(x>2.5) = P(3) + P(5) + P(8) = 0.4 + 0.3 + 0.1 = 0.8$

c) $P(2.7 < X < 5.1) = P(3) + P(5) = 0.4 + 0.3 = 0.7$

d) $E(x) = u_{x} = 2 0.2 + 3 0.4 + 5 0.3 + 8 0.1$ \\

$u_{x} = 3.9$

e) 

$Var(x) = (2-3.9)^{2}0.2 + (3-3.9)^{2} 0.4 + (5-3.9)^{2} 0.3 + (8 -3.9)^{2} 0.1 $

$Var(x) = 0.722 + 0,324 + 0,363 + 1,681$

$Var(x) = 3.09$

\end{solution}


\begin{problem}{6}
\textbf{[0.04pts]} Questao\_6 O  número  de  esquentadores  vendidos  diariamente  num  estabelecimento  é  bem descrito por uma variável aleatória com a seguinte função de probabilidade:

\begin{center}
\begin{tabular}{ |c|c|c|c|c|c|}
\hline

x & 0 & 1 & 2 & 3 & 4 \\
f(x) & a & b & c & b & a \\

 \hline
\end{tabular}
\end{center}


Se em vinte por cento dos dias as vendas são inferiores a uma unidade e, em  trinta por 
cento dos dias as vendas são superiores a duas unidades. 

a) Determine  as  constantes  a,  b,  e  c,  bem  como  a  função  distribuição  da  variável aleatória X. 

b) Determine  a  probabilidade  de,  quando  considerado  dois  dias,  as  vendas  sejam 
superiores em cada um deles, a uma unidade.

c) Se  cada  esquentador  é  vendido  a  75  euros,  determine  a  função  probabilidade  da receita bruta obtida com a venda de esquentadores num dia. 

d) Se num dia a receita bruta for inferior a 200 euros, determine a probabilidade de ser 
superior a 100 euros. 

\end{problem}

\begin{solution}

a) 

20\% dias vendas $< 1 P(X<1) = P(0) = 0.2$

30\% dias vendas $> 2 P(x>2) = P(3) + P(4) = 0.3$

P(0) = a = 0.2

P(3) = b = 0.1

P(0)+P(1)+P(2)+P(3)+P(4) = 1 $\newline$

O que implica: 2a+2b+c = 1 

c = 0.4

F(X=x) = 0   quando, $x<0$ 

F(X=x) = 0,2 quando, $0 \leq x < 1$ 

F(X=x) = 0,3 quando, $1 \leq x < 2$ 

F(X=x) = 0,7 quando, $2 \leq x < 3$ 

F(X=x) = 0.8 quando, $3 \leq x < 4$ 

F(X=x) = 1   quando, $x>4$ 

b) 

$P(X=1 \cap X=1) = P(X=1) \times P(X=1) = 0.7 \times 0.7 = 0.49$

c) 

\begin{center}
\begin{tabular}{ |c|c|c|c|c|c|c|}
\hline

x & 0 & 1 & 2 & 3 & 4 \\
r & 0,75 & 1,75 & 2,75 & 3,75 & 4,75 \\
r & 0 & 75 & 150 & 225 & 300 \\
P(r) & 20\% & 10\% & 40\% & 10\% & 20\% \\
 \hline
\end{tabular}
\end{center}

d) $P(\frac{R>100}{R<200}) = \frac{P(100<R<200)}{R<200}=\frac{0.4}{0.7} = 0.5714$

\end{solution}

\begin{problem}{7}
\textbf{[0.04pts]} Questao\_7 A percentagem de álcool em certo composto pode ser considerada uma v.a. X, onde 
$0 < x < 1$, com a f.d.p, $f(x)=x^{3}(1−x),0<x <1.$
\end{problem}

\begin{solution}

$f(x) = c x^{3}(1-x)$ quando, 0<x<1 f.d.p

a) Determine o valor de c

$\int_{0}{1} f(x) \, dx = 1$
$\int_{0}{1}  c x^{3}(1-x) \, dx = 1$
$(\frac{x^{4}}{4} - \frac{x^{5}}{5})|_{0}^{1} = 1/c$

$\frac{1}{4}-\frac{1}{5}=\frac{1}{c}$

$c=20$

b) Calcule $P(\frac{1}{2} < x < 1)$

$P(\frac{1}{2} < x < 1) = \int_{\frac{1}{2}}^{1} f(x) \,dx$
$P(\frac{1}{2} < x < 1) = 20 (\frac{x^{4}}{4} - \frac{x^{5}}{5})$

$P(\frac{1}{2} < x < 1) = 0.8125$

\end{solution}

\begin{problem}{8}
\textbf{[0.04pts]} Questao\_8 O diâmetro X de um cabo elétrico supõe-se ser uma v.a. com f.d.p, assim definida:

$f(x) = 6x(1-x)$, QUANDO, $0 \leq x < 1$
$f(x) = 0$, QUANDO, caso contrário

a) Verifique tratar-se de um f.d.p.

b)  Calcule $P(X \leq \frac{1}{2} | \frac{1}{3} < X < \frac{2}{3})$

\end{problem}

\begin{solution}

$f(x) = 6x(1-x)$ quando, $0 \leq x < 1$
$f(x) = 0$ caso contrário

a) Sim

b) $P(A|B) = \frac{P(AB)}{P(B)} = \frac{P(1/3 <x \leq 1/2}{P(1/3<x<2/3}$

$\int_{a}^{b} f(x) \,dx = 6 (\frac{x^{2}}{x^{3}/3}$

$P(1/3 < X \geq 1/2) = 6 ((\frac{1}{8} - \frac{1}{24}) -(\frac{1}{18} - \frac{1}{81})) = 0.2407 $

$P(1/3 < X \geq 1/2) = 6 ((\frac{4}{18} - \frac{8}{81}) -(\frac{1}{18} - \frac{1}{81})) = 0.4814 $

$P(x \geq 1/2 / 1/3 < X \geq 2/3) = 0.2407/0.4814 = \frac{1}{2}$

\end{solution}

\begin{problem}{9}
\textbf{[0.04pts]} Questao\_9 Suponha  que  uma  pequena  estação  de  serviço  é  abastecida  com  gasolina  todos  os 
sábados  à  tarde.  O  seu  volume  de  vendas,  em  milhares  de  litros,  é  uma  variável aleatória. Considerando que a função densidade de X é:

$f(x) = 6x(1-x)$, QUANDO, $0 \leq x < 1$

$f(x) = 0$, QUANDO, caso contrário

a) Determine $P(0 < X < 0.75$

b) Determine o volume médio de vendas e sua variância.

\end{problem}

\begin{solution}

a) 

$\int_{a}^{b} f(x) \, dx = 6 (\frac{x^{2}}{2} -\frac{x^{3}}{3}) = 0.84375$

b) 

$u_{x} = \int_{-\infty}^{\infty} xf(x) \,dx$
$u_{x} = \int_{0}^{1} 6(x^{2} - x^{3}) \,dx$

$u_{x} = 6(\frac{1}{3} - \frac{1}{4} = 6 \frac{1}{12} = \frac{1}{2}$

$u_{x} = 0.5$

$\sigma_{x}^{2} = \int_{0}^{1} (x-u_{x})^{2} f(x) \, dx$
$\sigma_{x}^{2} = \int_{0}^{1} (x-\frac{1}{2})^{2} f(x) \, dx$

$\sigma_{x}^{2} = \int_{0}^{1} 6 (x^{2} - x + \frac{1}{4}) (x-x^{2}) \, dx$

$\sigma_{x}^{2} = 6 (\frac{x^{4}}{4}-\frac{x^{3}}{3}+\frac{x^{2}}{8}-\frac{x^{5}}{5}+\frac{x^{4}}{4}-\frac{x^{3}}{12}$

$\sigma_{x}^{2} = 0.05$

\end{solution}

\begin{problem}{10}
\textbf{[0.04pts]} Questao\_10 A  Procura  mensal  de  um  certo  artigo  é  uma  variável  aleatória  X  com  a  seguinte 
função densidade:

$f(x) = \frac{x}{4}$, QUANDO, $0 \leq x \leq 2$

$f(x) = 1-\frac{x}{4}$, QUANDO, $2 \leq x \leq 4$

$f(x) = 0$, QUANDO, caso contrário

a) Calcule a funçao de distribuicao de x



\end{problem}

\begin{solution}

a) Calcule a funçao de distribuicao de x

$F(X) = \int_{0}^{x} \frac{x}{4} \, dx$

$F(x) = \frac{x^{2}}{8} 0 \geq x < 2$

$F(x) = \int_{2}^{x} 1-\frac{x}{4} \, dx + \int_{0}^{2} \frac{x}{4} \, dx = (x - \frac{x^{2}}{8}) + \frac{x^{2}}{8}$

$F(x) = x - \frac{x^{2}}{8}-2+\frac{4}{8}+\frac{4}{8}$

$F(x) = x - \frac{x^{2}}{8}-1$ quando $2 \leq x < 4$

$F(x) = 0$                   quando, $x <0 $
$F(x) = \frac{x^{2}}{8}$     quando, $0 \leq x <2 $
$F(x) = x-\frac{x^{2}}{8}-1$ quando, $2 \leq x <4 $
$F(x) = 1$                   quando, $x >4 $

\end{solution}

\begin{problem}{11}
\textbf{[0.04pts]} Questao\_11 O tempo de permanência dos alunos numa aula de 2 horas é uma variável aleatória X com densidade de probabilidade: 


$f(x) = 1-\frac{x}{2}$, QUANDO, $0 \leq x \leq 2$

$f(x) = 0$, caso contrário

a) Qual a probabilidade de um aluno, escolhido ao acaso, assistir a mais de 75\% da aula ?

b) De  10  alunos,  escolhidos  ao  acaso  no  conjunto  dos  alunos  que  estão  presentes  no início  da  aula,  qual  a  probabilidade  de  apenas  1  permanecer  na  sala  quando  faltam  15 minutos para a aula terminar. 

\end{problem}

\begin{solution}

a) $P[x>75\%]$

Aula_Total = 2horas o que quer dizer que 75\% 2hrs = 1.5

$P(X > 1.5) = \int_{1.5}^{2} (1-\frac{x}{2} \,dx = (x-\frac{x^{2}}{4})$

$P(X > 1.5) = 0.0625$

b) 


\end{solution}

\begin{problem}{12}
\textbf{[0.04pts]} Questao\_12 Considere a tabela que representa a função massa de probabilidade conjunta do par 
aleatório (X,Y): 

\begin{center}
\begin{tabular}{ |c|c|c|c|c|c|}
\hline

$X|Y$ & 0 & 1 & 2 & 3  \\
1 & 0,1 & 0.15 & 0.2 & p  \\
2 & P & 0.15 & 0.15 & 0.05 \\
3 & 0.05 & 0 & P & 0  \\
 \hline
\end{tabular}
\end{center}

a) Determine o valor de p e obtenha as funções massa de probabilidade marginal X e Y.

b) Defina a função distribuição da variável aleatória X.

c) Verifique se X e Y são variáveis aleatórias independentes. Calcule a covariância das variáveis X e Y.

d) Considere a  variável  aleatória W=X+2Y.  Calcule  o  seu  valor  médio e  sua variância.

e) Calcule $P(X+Y \leq 3 | Y_{impar}$ 

\end{problem}

\begin{solution}


Função massa de probabilidade = função de densidade de probabilidade (v.a. continuas)

a) Determine o valor de p e fmp de x y

$0.15 + 0.30 + 0.35 + 0.05 + 3p = 1$
$0.85 + 3p = 1$

$3p = 0.15$ 
$p = 0.05$


\begin{center}
\begin{tabular}{ |c|c|c|c|}
\hline

X & 1 & 2 & 3 \\
P(x) & 0.5 & 0.4 & 0.1 \\

 \hline
\end{tabular}
\end{center}

\begin{center}
\begin{tabular}{ |c|c|c|c|c|}
\hline

Y & 0 & 1 & 2 & 3  \\
P(y) & 0.2 & 0.3 & 0.4 & 0.1 \\

 \hline
\end{tabular}
\end{center}

b) Defina a função distribuicao da v.a. X

$F(x) = 0 $ quando, $x<1$
$F(x) = 0.5 $ quando, $x<1$
$F(x) = 0.9 $ quando, $x<1$
$F(x) = 1 $ quando, $x \geq 3$

c) Não são independentes porque P(x) depende da variável de Y

Calculo da variância:

$\sigma_{xy} = E(xy) - u_{x}u_{y}$

$u_{x} = 1 0.5 + 2 0.4 + 3 0.1$
$u_{x} = 0.5 + 0.8 + 0.3 = 1.6$

$u_{y} = 0 + 0.3 + 0.8 + 0.3$
$u_{y} = 1.4$

$E[xy] = \sum_{0}^{3}\sum_{1}^{3} x_{i}y_{i} P(x_{i}y_{i})$

$E[xy] = 0 + 0.15 + 0.30 + 2 (0.2+0.3+0.12)+3(0.05+0.1)$
$E[xy] = 0.45 + 1.30 + 0.45$
$E[xy] = 2.24$

$\sigma_{xy} = 2.20 - (1.6 \times 1.4)$
$\sigma_{xy} = -0.04$

$w= x + 2Y$

\begin{center}
\begin{tabular}{ |c|c|c|c|c|c|c|c|c|c|}
\hline

w & 1 & 2 & 3 & 4 & 5 & 6 & 7 & 8 & 9 \\
P(w) & 0.1 & 0.05 & 0.20 & 0.15 & 0.20 & 0.15 & 0.1 & 0.05 & 0 \\

 \hline
\end{tabular}
\end{center}

$E[W] = u_{w} = 0.1 + 0.1 + 0.6 + 0.6 + 1 + 0.9 + 0.7 + 0.4$
$u_{w} = 4.4$

$Var[w] = \sigma_{w}^{2} = E((x-u_{x})^{2}) = \sum_{1}^{9} (w_{i} - u_{w})^2 p(w_{i})$

$\sigma_{w}^{2} = (1-4,4)^{2} 0.1 + (2-4.4)^{2} 0.05 + (3-4.4)^{2} 0.2 + (4-4.4)^{2} 0.15 + (5-4.4)^{2} 0.2 + (6-4.4)^{2} 0.15 + (7-4.4)^{2} 0.1 + (8-4.4)^{2} 0.05$

$\sigma_{w}^{2} = Var(w) = 3.64$

d) Calcule $P(x+y \leq 3 | $ y é impar

$P(Y_{impar}) = P(1,1) + P(1,3) + P(2,1) + P(2,3) + P(3,1) + P(3,3)$

$P(Y_{impar}) = 0.15 + 0.05 + 0.15 + 0.05 + 0 + 0 = 0.4$

$P(x+y \leq 3 | Y_{impar}) = 0.15 + 0.15 = 0.30$

$P(x+y \leq 3 | Y_{impar}) = 0.3/0.4 = 0.75$


\end{solution}

\begin{problem}{13}
\textbf{[0.04pts]} Questao\_13 Numa loja de eletrodomésticos as vendas mensais de frigoríficos das marcas A e B 
são variáveis aleatórias X e Y com distribuição conjunta dada por:

\begin{center}
\begin{tabular}{ |c|c|c|c|c|}
\hline

$X|Y$ & 1 & 2 & 3  \\
1 & 0.25 & 0.25 & 0 \\
2 & 0 & 0.25 & 0.25 \\

 \hline
\end{tabular}
\end{center}

a) Calcule  a  probabilidade  de  em  um  dado  mês  se  vender  um  número  idêntico  de frigoríficos das marcas A e B.

b) Determine as probabilidades marginais de X e Y.

c)  Determine  a  covariância  das  variáveis  X  e  Y.  Que  conclusão  pode  retirar  acerca  da independência entre as vendas da marca A e B?

d) Seja Z o total das vendas mensais das marcas A e B. Calcule a variância Z.

e) Sabe-se  que  foi  vendido  um  frigorífico  da  marca  A.  Qual  a  distribuição  da  variável aleatória $Y|X = 1$?

\end{problem}

\begin{solution}



a) $P[A=B] = P(1,1) + P(2,2) = 0,25 + 0,25 = 0,5$

b) Probabilidade marginal:

$P(X=x)$
$P(1,1) + P(1,2) + P(1,3) = 0.5$
$P(2,1) + P(2,2) + P(2,3) = 0.5$

$P(Y=y)$
$P(1,1) + P(1,2)  = 0.25$
$P(2,1) + P(2,2)  = 0.50$
$P(3,1) + P(3,2)  = 0.25$

c) Determine a covariância das v.a. X e Y

$E[xy] = 1 0.25 + 2 0.25 + 4 0.25 + 0 + 6 0.25 = 3.25$

$u_{x} = 1 0.5 + 2 0.5 = 1.5$
$u_{y} = 1 0.25 + 2 0.5 + 3 0.25 = 2$

$\sigma_{xy} = 3.25 - 1.5 2 = 0.25$

\textbf{não são indepemdentes}

d) $Z=A+B$

$Var(z) = \sigma_{z}^{2} = (2-3.5)^{2} 0.25 + (3-3.5)^{2} 0.25 + (4-3.5)^{2} 0.25 + (5-3.5)^{2} 0.25$

$\sigma_{z}^{2} = 0.5625 + 0.0625 + 0.0625 + 0.5625$
$\sigma_{z}^{2} = 1.25$

e) A=X=1

$P(x=1) = 0.5$

$P(1,1) = 0.25$
$P(1,2) = 0.25$
$P(1,3) = 0$

Assim,

$P(1,1)/P_{x}(1) = 0.5$
$P(1,2)/P_{x}(1) = 0.5$
$P(1,3)/P_{x}(1) = 0$

$P(Y=y/x=1) = 0.5$ QUANDO, y=1
$P(Y=y/x=1) = 0.5$ QUANDO, y=2
$P(Y=y/x=1) = 0$ QUANDO, y=3

\end{solution}

\begin{problem}{14}
\textbf{[0.04pts]} Questao\_14 Uma  agência  de  um  banco  estudou  as  duas  variáveis  seguintes  com  o  objetivo  de conhecer o comportamento dos titulares de conta corrente: 

X -“Número de produtos do banco utilizados além da conta corrente” 

Y -“Número de meses com a conta corrente a descoberto no ano anterior” 

A distribuição conjunta destas variáveis é indicada na tabela seguinte: 

\begin{center}
\begin{tabular}{ |c|c|c|c|}
\hline

$X|Y$ & 0 & 1 & 2  \\
0 & 0.3 & 0.1 & 0.05 \\
1 & 0.2 & 0.1 & 0.05 \\
2 & 0.05 & 0.05 & 0 \\
3 & 0.1 & 0 & 0 \\

 \hline
\end{tabular}
\end{center}

a) Determine as probabilidades marginais de X e Y.

b) Qual  a  probabilidade  de  um  cliente  titular  de  conta  corrente  possuir  produtos  do banco além da conta corrente.

c) Sabendo  que  um  dado  cliente  apresentou  a  conta  a  descoberto  em  mais  do  que  um mês do ano passado, qual a probabilidade desse cliente utilizar produtos do banco além da conta corrente?

d) Qual  o  número  médio  de  meses  que  um  cliente  da  agência  apresentou  a  conta  a descoberto no ano anterior? Com que desvio padrão?

\end{problem}

\begin{solution}

a) Probabilidades marginais

$P(X=x) = 0.45$ quando, x=0
$P(X=x) = 0.35$ quando, x=1
$P(X=x) = 0.1$  quando, x=2
$P(X=x) = 0.1$  quando, x=3

P(Y=y) = 0.65 quando y=0
P(Y=y) = 0.25 quando y=1
P(Y=y) = 0.1 quando y=2

b) $P(x>0) = P_{x}(1) + P_{x}(2) + P_{x}(3) = 0.35 + 0.10 + 0.10$

$P(x>0) = 0.55$

c) P(x \geq 1 / y >1) = \frac{P(1,1) + P(1,2) + P(1,3)}{P_{y}(2)} = \frac{0.05}{0.1} = 0.5

d) $u_{y} = 0 0.65 + 1 0.25 + 2 0.1 = 0.25 + 0.2 = 0.45$

$\sigma_{y}^{2} = (0 - 0.45) 0.65 + (1-0.45) 0.25 + (2-0.45) 0.1$

$\sigma_{y}^{2} = 0.4475$

Assim, $\sqrt{\sigma_{y}^{2} =} = 0.669$


\end{solution}

\begin{problem}{15}
\textbf{[0.04pts]} Questao\_15 Um  comerciante  vende,  entre  outros  artigos,  máquinas  de  calcular  de  certo  tipo. Seja X a variável aleatória que representa o número de máquinas de calcular disponíveis para venda no início de cada semana. O comerciante possui uma dessas máquinas para 
seu próprio uso, que poderá eventualmente vender, mas que não conta como disponível. O  número  de  máquinas  vendidas  semanalmente  é  uma  variável  aleatória  Y.  O  par 
aleatório (X,Y) apresenta a seguinte distribuição de probabilidades:

a) Determine  a  percentagem  de  semanas  em  que  se  vendem  mais  máquinas  do  que  a média das máquinas disponíveis.

\end{problem}

\begin{solution}

$P(X=i, Y=i) = \frac{1}{12} i=1,2,3 0 \leq j \leq i+1 $

a) Vende mais máquinas do que a média disponível

\begin{center}
\begin{tabular}{ |c|c|c|c|c|c|c|}
\hline

$X|Y$ & 0 & 1 & 2 & 3 & 4 & soma\\
1 & 1/12 & 1/12 & 1/12 & 0 & 4 & 3/12 \\
2 & 1/12 & 1/12 & 1/12 & 1/12 & 4 & 4/12 \\
3 & 1/12 & 1/12 & 1/12 & 1/12 & 4 & 5/12 \\
soma & 3/12 & 3/12 & 3/12 & 2/12 & 1/12 & 1  \\
 \hline
\end{tabular}
\end{center}

a) $u_{x} = 1 3/12 + 2 4/12 + 3 5/12 = 26/12 = 2.17$

$P_{y}(y>2.17) = P_{y}(3) + P_{y}(4) = 2/12 + 1/12 = 3/12 = 0.25$

b) $P(x>y) = P(1,0) + P(2,1) + P(3,2) = \frac{1}{12}+\frac{1}{12}+\frac{1}{12} = \frac{3}{12}$

c) $P(X=Y-1) = P(1,2) + P(2,3) + P(3,4) = \frac{3}{12}$




\end{solution}

\begin{problem}{16}
\textbf{[0.04pts]} Questao\_16 Determine  a  função  de  probabilidade  para  a  variável  aleatória  com  a  seguinte função de distribuição cumulativa:

F(x) = 0,   QUANDO, x <2
F(x) = 0.2, QUANDO, 2≤ x <5.7
F(x) = 0.5, QUANDO, 5.7≤ x <6.5
F(x) = 0.8, QUANDO, 6.5≤ x <8.5
F(x) = 1,   QUANDO, 8.5≤ x

\end{problem}

\begin{solution}

f(0) = 0
f(2) = 0.2 - f(0) = 0.2
f(5.7) = 0.5 - f(2) = 0.5 - 0.2 = 0.3
f(6.5) = 0.8 - f(5.7) - f(2) = 0.8 - 0.3 - 0.2 = 0.3
f(8.5) = 1 - f(6.5) - f(5.7) - f(2) = 1-0.3-0.3-0.2 = 0.2

Assim,

f(X) = 0,   QUANDO, x=0
f(x) = 0.2, QUANDO, x=2
f(x) = 0.3, QUANDO, x=5.7
f(x) = 0.3, QUANDO, x=6.5
f(x) = 0.2, QUANDO, x=8.5

\end{solution}

\begin{problem}{17}
\textbf{[0.04pts]} Questao\_17 Uma variável aleatória X é normalmente distribuída com média igual 1200 e desvio 
padrão igual a 200. Qual a probabilidade P[800≤ x ≤1000]
\end{problem}

\begin{solution}

$X = [1200, 200] = [0,1]$
$P[800 \leq x \leq 1000] = P(\frac{800-1200}{200} < z < \frac{1000 - 1200}{200}$

$P(-2 \leq z \leq -1)$

Assim,

$P[-2 \leq x \leq -1] = Z(-1) - Z(-2) = Z(2) - Z(1) = 0.4772 - 0.3413 = 0.1359$

\end{solution}

\begin{problem}{18}
\textbf{[0.04pts]} Questao\_18 Uma variável aleatória X associada a função densidade de probabilidade

$f(x) = \frac{1}{20}e^{-\frac{x}{20}}$ quando, $x \geq 0$

$f(x) = 0 $ caso contrário

Determine a probabilidade de X assumir valores entre 10 e 20.

\end{problem}

\begin{solution}

$P(10 \leq X \leq 20) = \int_{10}^{20} f(x) \, dx$

$\int_{10}^{20} \frac{1}{20} e^{-\frac{x}{20}} \, dx$ \\
$P(10 \leq X \leq 20) = (-e^{-\frac{x}{20}})_{10}^{20}$

$P(10 \leq X \leq 20) = (-e^{-1} + e^{-\frac{1}{2}})$

\end{solution}

\begin{problem}{19}
\textbf{[0.04pts]} Questao\_19 Considere  o  lançamento  de  dois  dados  e  a  experiência  cujo  resultado  consiste  na soma  do  número  de  pontos  das  faces  que  resultam  do  lançamento.  Defina  esta  soma 
como sendo uma variável aleatória X. Esboce a função distribuição de probabilidade da v.a. X.
\end{problem}

\begin{solution}

\begin{center}
\begin{tabular}{ |c|c|c|c|c|c|c|}
\hline

$\frac{d1}{d2}$ & 1 & 2 & 3 & 4 & 5 & 6  \\
1 & 2 & 3 & 4 & 5 & 6 & 7 \\
2 & 3 & 4 & 5 & 6 & 7 & 8 \\
3 & 4 & 5 & 6 & 7 & 8 & 9 \\
4 & 5 & 6 & 7 & 8 & 9 & 10 \\
5 & 6 & 7 & 8 & 9 & 10 & 11 \\
6 & 7 & 8 & 9 & 10 & 11 & 12 \\
 \hline
\end{tabular}
\end{center}

$X= d1 + d2$

\begin{center}
\begin{tabular}{ |c|c|c|c|c|c|c|c|c|c|c|c|}
\hline

$x_{i}$ & 2 & 3 & 4 & 5 & 6 & 7 & 8 & 9 & 10 & 11 & 12  \\
$P(X=x_{i})$ & $\frac{1}{36}$ & $\frac{2}{36}$ & $\frac{3}{36}$ & $\frac{4}{36}$ & $\frac{5}{36}$ & $\frac{6}{36}$ & $\frac{5}{36}$ & $\frac{4}{36}$ & $\frac{3}{36}$ & $\frac{2}{36}$ & $\frac{1}{36}$  \\

 \hline
\end{tabular}
\end{center}

F(X) = 0,x <2 \\
F(X) = 1/36,  QUANDO, $x =2$ \\
F(X) = 3/36,  QUANDO, $x\leq3$ \\
F(X) = 6/36,  QUANDO, $x\leq4$ \\
F(X) = 10/36, QUANDO, $x\leq5$ \\
F(X) = 15/36, QUANDO, $x\leq6$ \\
F(X) = 21/36, QUANDO, $x\leq7$ \\
F(X) = 26/36, QUANDO, $x\leq8$ \\
F(X) = 30/36, QUANDO, $x\leq9$ \\
F(X) = 33/36, QUANDO, $x\leq10$ \\
F(X) = 35/36, QUANDO, $x\leq11$ \\
F(X) = 1,     QUANDO, $x\leq12$ \\

\end{solution}

\begin{problem}{20}
\textbf{[0.04pts]} Questao\_20  A  função  distribuição  de  probabilidade  de  uma  variável  aleatória,  X  tem a forma apresentada na figura a seguir: 

a) Determine o valor Fx(0), a probabilidade de uma v.a.

b) $P(X \geq 0)$ 

\end{problem}

\begin{solution}

$a = \frac{\Delta y}{\Delta x} = \frac{1}{10}$

$b = y - \frac{1}{10}x$

$b = 1 - \frac{1}{10} \times 5 = \frac{1}{2}$

$F_{x}(0) = \frac{1}{2}$

b) $P(X \geq 0)$

$F(X) = \frac{1}{10}x + \frac{1}{2}$ quando, $-5 \leq x < 5$

$F(X) = 1$ quando $x \geq 5$

$F(5) - F(0) = 1 - \frac{1}{2} = \frac{1}{2}$


\end{solution}

\begin{problem}{21}
\textbf{[0.04pts]} Questao\_21 Determine  o  valor  de  a  para  que  a  função  dada  abaixo  possa  representar  uma função densidade de probabilidade contínua. 

$f(x,y) = a(x^{2} + \frac{xy}{3})$ quando, $0 \leq x \leq 1$ ; $0 \leq y \leq 2$

$f(x,y) = 0$ caso contrário

\end{problem}

\begin{solution}

$\int_{0}^{1} \int_{0}^{2} a(x^{2} + \frac{xy}{3}) \,dy \,dx = 1$

$a \int_{0}^{1} (x^{2}y + \frac{xy^{2}}{6})_{0}^{2} \,dx = 1$

$a (\frac{2x^{3}}{3} + \frac{1x^{2}}{3})_{0}^{1} = 1$

$ a=1$

\end{solution}

\begin{problem}{22}
\textbf{[0.04pts]} Questao\_22 Considerando o exemplo anterior e o valor de a obtido determine:

$P(x \leq \frac{1}{2} \leq 1)$

\end{problem}

\begin{solution}

$\int_{0}^{\frac{1}{2}} \int_{0}^{1} x^{2} + \frac{xy}{3} \, dy \, dx$

$\int_{0}^{\frac{1}{2}} x^{2} + \frac{x}{6} \, dx = \frac{1}{24}+\frac{1}{48} = \frac{3}{48} = \frac{1}{16}$

\textbf{Resposta: $\frac{1}{16}$}



\end{solution}

\begin{problem}{23}
\textbf{[0.04pts]} Questao\_23 Sabendo-se que a função densidade de probabilidade conjunta é dada por:

$f(x,y) = a(x^{2} + \frac{xy}{3})$ quando, $0 \leq x \leq 1$ ; $0 \leq y \leq 2$

Determine  a  função  densidade  de  probabilidade  marginal  e  a  função  distribuição  de 
probabilidade marginal para a variável aleatória X:
\end{problem}

\begin{solution}

$f_{x}(x) = \int_{0}^{2} x^{2}y + \frac{xy}{6} \, dx$

Assim,

$f_{x}(x) = 2x^{2} + \frac{2x}{3}$ QUANDO, $0 \leq x \leq 1$

$f_{x}(x) = 0$ caso contrário

Calcularemos os intervalos:

x < 0

$F(x) = \int_{-\infty}^{0} 2x^{2} + \frac{2x}{3} \, dx = 0$

$F(x) = 0$

$0 \leq x \leq 1$

$F(x) = \int_{0}^{x} 2x^{2} + \frac{2x}{3} \, dx = 0$

$F(x) = (\frac{2x^{3}}{3} + \frac{2x^{2}}{6})_{0}^{x}$

$F(x) = \frac{2x^{3}}{3}+\frac{x^{3}}{3}$

x > 1

$F(x) = \int_{-\infty}^{0} 2x^{2} + \frac{2x}{3} \, dx = 1$


\end{solution}

\begin{problem}{24}
\textbf{[0.04pts]} Questao\_24 Considerando  que  a  função  densidade  de  probabilidade  conjunta  das  variáveis aleatórias é dada por:

Determine: 
a) Função densidade de probabilidade condicional $f_{x|y}(X|Y)$

b) $f_{X|y}(x|y) = P(X \leq x, Y=y)$

\end{problem}

\begin{solution}

a) $f_{x|y}(X|Y) = \frac{f(x,y)}{f(y)}$

$f_{x|y}(X|Y) = \frac{\frac{x+y}{3}}{\int_{0}^{1}\frac{x+y}{3} \,dx}$

$f_{x|y}(X|Y) = \frac{x+y}{3} \frac{6}{2y+1} = \frac{x+y}{y+0.5}$ QUANDO, $0 \leq x \leq 1 ; 0 \leq y \leq 2$

b) $F_{x|y}(x|y) = \frac{\int_{-\infty}^{x}f(x,y) \, dx}{\int_{0}^{1} f(x,y) \, dx}$

$F_{x|y}(x|y) = \frac{(\frac{x+y}{y+0.5})_{0}^{x}}{(\frac{x+y}{y+0.5})_{0}^{1}} $

$F_{x|y}(x|y) = \frac{\frac{x+y}{y+0.5}}{\frac{1}{2}+y}$

$f(x,y) = 0 $, QUANDO $x<0;y<0$

$f(x,y) = \frac{\frac{x^{2}}{2}+xy}{\frac{1}{2}+y} 
$, QUANDO $0 \leq x \leq 1; 0 \leq y \leq 2$

$f(x,y) = 1 $, caso contrário

\end{solution}

\begin{problem}{25}
\textbf{[0.04pts]} Questao\_25 Verifique se as variáveis aleatórias X e Y são independentes, sabendo-se que a função densidade de probabilidade conjunta é dada por: 

\end{problem}

\begin{solution}

$f(x,y) = \frac{1}{2} e^{-y}$ quando, $0 \leq x \leq 2, y> 0$

$f_{x}(x) = \int_{0}^{\infty} \frac{1}{2} e^{-y} \, dy$

$f_{x}(x) = -\frac{1}{2} e^{-y}|_{0}^{\infty} = 0 + \frac{1}{2} = \frac{1}{2}$

$f_{y}(y) = \int_{0}^{2} \frac{1}{2} e^{-y} \, dy$

$f_{y}(y) = \frac{1}{2} e^{-y}x|_{0}^{2} = (\frac{e^{-y}}{2} \times 2) - (\frac{e^{-y}}{2} \times 0) = e^{-y}$

$f(x,y) = f_{x}(x) \times f_{y}(y) = \frac{e^{-y}}{2} \newline$

Assim, é observado que os eventos são independentes

\end{solution}

\end{document}





