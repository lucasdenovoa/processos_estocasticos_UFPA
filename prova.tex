%%%%%%%%%%%%%%%%%%%%%%%%%%%%%%%%%%%%%%%%%%%%%%%%%%%%%%%%%%%%%%%%%%%%%%%%%%%%%%%%%%%%
% Do not alter this block (unless you're familiar with LaTeX
\documentclass{article}
\usepackage[margin=1in]{geometry} 
\usepackage{amsmath,amsthm,amssymb,amsfonts, fancyhdr, color, comment, graphicx, environ}
\usepackage{xcolor}
\usepackage{mdframed}
\usepackage[shortlabels]{enumitem}
\usepackage{indentfirst}
\usepackage{hyperref}
\hypersetup{
    colorlinks=true,
    linkcolor=blue,
    filecolor=magenta,      
    urlcolor=blue,
}


\pagestyle{fancy}


\newenvironment{problem}[2][Problem]
    { \begin{mdframed}[backgroundcolor=gray!20] \textbf{#1 #2} \\}
    {  \end{mdframed}}

% Define solution environment
\newenvironment{solution}
    {\textit{Solução:}}
    {}

\renewcommand{\qed}{\quad\qedsymbol}

% prevent line break in inline mode
\binoppenalty=\maxdimen
\relpenalty=\maxdimen

%%%%%%%%%%%%%%%%%%%%%%%%%%%%%%%%%%%%%%%%%%%%%
%Fill in the appropriate information below
\lhead{Your name: Lucas de Nóvoa Martins Pinto}

\chead{\textbf{Homework 2  Due: 17 June 2019 at beginning at class}}
%%%%%%%%%%%%%%%%%%%%%%%%%%%%%%%%%%%%%%%%%%%%%

\begin{document}

\begin{mdframed}[backgroundcolor=blue!20]
Trabalho processo estocasticos
\end{mdframed}

\begin{problem}{1}
\textbf{[0,037pts]} Questao\_01 Determinar a probabilidade de obtenção de uma cara e duas coroas em 3 arremessos de uma moeda ideal.

\end{problem}

\begin{solution}

C - Cara

K - Coroa

Temos 3 possibilidades possiveis sendo: CKK - KCK \textbf{OU} KKC

Cada uma dessas tem a chance de:

\[\frac{1}{2} * \frac{1}{2}  * \frac{1}{2} = \frac{1}{8}\] de ocorrer

assim:

\[\frac{1}{8} + \frac{1}{8}  + \frac{1}{8} = \frac{3}{8}\]

\end{solution}

\begin{problem}{2}
\textbf{[0,037pts]} Questao\_2 Uma urna contém cinco bolas numeradas. Suponha que selecionamos duas bolas da urna com reposição. Quantos pares ordenados distintos são possíveis? Qual é a probabilidade de retirar duas vezes a mesma bola?

\end{problem}

\begin{solution}

Dado, um par ordenado como sendo (a, b)

Se tem reposição então o espaço amostral é o mesmo e é igual:
\[ (1,1) (1,2) (1,3) (1,4) (1,5) = 5 \newline \]
\[ (2,1) (2,2) (2,3) (2,4) (2,5) = 5 \newline \]
\[ (3,1) (3,2) (3,3) (3,4) (3,5) = 5 \newline \]
\[ (4,1) (4,2) (4,3) (4,4) (4,5) = 5 \newline \]
\[ (5,1) (5,2) (5,3) (5,4) (5,5) = 5 \newline \]

entao, 5 \times 5 = \textbf{25 Resultados possíveis}


%Sempre que for colocar alguma formular usar o $$
b) A probabilidade de tirar uma bola e $\frac{1}{5}$

Assim, $\frac{1}{5} \times \frac{1}{5} = \frac{1}{25}$

Agora observe que podemos repetir isso 5 vezes: $\frac{1}{25} \times 5 = 0,2$

\end{solution}

\begin{problem}{3}
\textbf{[0,037pts]} Questao\_3 Uma urna contém cinco bolas numeradas. Suponha que selecionamos duas bolas da urna em sucessão, e sem reposição. Quantos pares ordenados distintos são possíveis? Qual é a 
probabilidade de que a primeira bola tenha um número maior que a segunda? 

\end{problem}

\begin{solution}

Sabemos que um par ordenado é dado como sendo (a, b)

Se tem reposição então o espaço amostral é o mesmo e é igual:

\[ (1,2) (1,3) (1,4) (1,5) = 4 \newline \]
\[ (2,1) (2,3) (2,4) (2,5) = 4 \newline \]
\[ (3,1) (3,2) (3,4) (3,5) = 4 \newline \]
\[ (4,1) (4,2) (4,3) (4,5) = 4 \newline \]
\[ (5,1) (5,2) (5,3) (5,4) = 4 \newline \]

entao, 5 \times 4 = \textbf{20 Resultados possíveis} \newline


b) A probabilidade de tirar uma bola e ela ser maior que a anterior ou seja $b>a$ é: \newline

$\frac{1}{5} \times \frac{1}{4} = \frac{1}{20}$ Para o caso de que o primeiro retirado foi \textbf{1}\newline
$\frac{1}{5} \times \frac{2}{4} = \frac{2}{20}$ Para o caso de que o primeiro retirado foi \textbf{2}\newline
$\frac{1}{5} \times \frac{3}{4} = \frac{3}{20}$ Para o caso de que o primeiro retirado foi \textbf{3}\newline
$\frac{1}{5} \times \frac{4}{4} = \frac{4}{20}$ Para o caso de que o primeiro retirado foi \textbf{4}\newline

Observe que o caso que o primeiro retirado é 5 não é válido porque não tem numero maior que esse\newline

Assim, vamos somar todas essas frações e teremos $\frac{10}{20} = 0.5$
\end{solution}

\begin{problem}{4}
\textbf{[0,037pts]} Questao\_4 Suponha que uma moeda é jogada três vezes. Se assumimos que as jogadas são independentes e a probabilidade de caras é p, encontre a probabilidade dos eventos nenhuma coroa, uma coroa, duas coroas e três coroas. 

\end{problem}

\begin{solution}


\end{solution}

\end{document}