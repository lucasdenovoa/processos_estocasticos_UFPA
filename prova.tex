%%%%%%%%%%%%%%%%%%%%%%%%%%%%%%%%%%%%%%%%%%%%%%%%%%%%%%%%%%%%%%%%%%%%%%%%%%%%%%%%%%%%
% Do not alter this block (unless you're familiar with LaTeX
\documentclass{article}
\usepackage[margin=1in]{geometry} 
\usepackage{amsmath,amsthm,amssymb,amsfonts, fancyhdr, color, comment, graphicx, environ}
\usepackage{xcolor}
\usepackage{mdframed}
\usepackage[shortlabels]{enumitem}
\usepackage{indentfirst}
\usepackage{hyperref}
\hypersetup{
    colorlinks=true,
    linkcolor=blue,
    filecolor=magenta,      
    urlcolor=blue,
}


\pagestyle{fancy}


\newenvironment{problem}[2][Questão]
    { \begin{mdframed}[backgroundcolor=gray!20] \textbf{#1 #2} \\}
    {  \end{mdframed}}

% Define solution environment
\newenvironment{solution}
    {\textit{Solução:}}
    {}

\renewcommand{\qed}{\quad\qedsymbol}

% prevent line break in inline mode
\binoppenalty=\maxdimen
\relpenalty=\maxdimen

%%%%%%%%%%%%%%%%%%%%%%%%%%%%%%%%%%%%%%%%%%%%%
%Fill in the appropriate information below
\lhead{Lucas de Nóvoa Martins Pinto}

\chead{\textbf{Homework 2  Due: 17 June 2019 at beginning at class}}
%%%%%%%%%%%%%%%%%%%%%%%%%%%%%%%%%%%%%%%%%%%%%

\begin{document}

\begin{mdframed}[backgroundcolor=blue!20]
Trabalho processo estocasticos
\end{mdframed}

\begin{problem}{1}
\textbf{[0,037pts]} Questao\_01 Determinar a probabilidade de obtenção de uma cara e duas coroas em 3 arremessos de uma moeda ideal.

\end{problem}

\begin{solution}

C - Cara

K - Coroa

Temos 3 possibilidades possiveis sendo: CKK - KCK \textbf{OU} KKC

Cada uma dessas tem a chance de:

\[\frac{1}{2} * \frac{1}{2}  * \frac{1}{2} = \frac{1}{8}\] de ocorrer

assim:

\[\frac{1}{8} + \frac{1}{8}  + \frac{1}{8} = \frac{3}{8}\]

\end{solution}

\begin{problem}{2}
\textbf{[0,037pts]} Questao\_2 Uma urna contém cinco bolas numeradas. Suponha que selecionamos duas bolas da urna com reposição. Quantos pares ordenados distintos são possíveis? Qual é a probabilidade de retirar duas vezes a mesma bola?

\end{problem}

\begin{solution}

Dado, um par ordenado como sendo (a, b)

Se tem reposição então o espaço amostral é o mesmo e é igual:
\[ (1,1) (1,2) (1,3) (1,4) (1,5) = 5 \newline \]
\[ (2,1) (2,2) (2,3) (2,4) (2,5) = 5 \newline \]
\[ (3,1) (3,2) (3,3) (3,4) (3,5) = 5 \newline \]
\[ (4,1) (4,2) (4,3) (4,4) (4,5) = 5 \newline \]
\[ (5,1) (5,2) (5,3) (5,4) (5,5) = 5 \newline \]

entao, $5 \times 5 = \textbf{25 Resultados possíveis}$


%Sempre que for colocar alguma formular usar o $$
b) A probabilidade de tirar uma bola e $\frac{1}{5}$

Assim, $\frac{1}{5} \times \frac{1}{5} = \frac{1}{25}$

Agora observe que podemos repetir isso 5 vezes: $\frac{1}{25} \times 5 = 0,2$

\end{solution}

\begin{problem}{3}
\textbf{[0,037pts]} Questao\_3 Uma urna contém cinco bolas numeradas. Suponha que selecionamos duas bolas da urna em sucessão, e sem reposição. Quantos pares ordenados distintos são possíveis? Qual é a 
probabilidade de que a primeira bola tenha um número maior que a segunda? 

\end{problem}

\begin{solution}

Sabemos que um par ordenado é dado como sendo (a, b)

Se tem reposição então o espaço amostral é o mesmo e é igual:

\[ (1,2) (1,3) (1,4) (1,5) = 4 \newline \]
\[ (2,1) (2,3) (2,4) (2,5) = 4 \newline \]
\[ (3,1) (3,2) (3,4) (3,5) = 4 \newline \]
\[ (4,1) (4,2) (4,3) (4,5) = 4 \newline \]
\[ (5,1) (5,2) (5,3) (5,4) = 4 \newline \]

entao, $5 \times 4 = \textbf{20 Resultados possíveis} \newline$


b) A probabilidade de tirar uma bola e ela ser maior que a anterior ou seja $b>a$ é: \newline

$\frac{1}{5} \times \frac{1}{4} = \frac{1}{20}$ Para o caso de que o primeiro retirado foi \textbf{1}\newline
$\frac{1}{5} \times \frac{2}{4} = \frac{2}{20}$ Para o caso de que o primeiro retirado foi \textbf{2}\newline
$\frac{1}{5} \times \frac{3}{4} = \frac{3}{20}$ Para o caso de que o primeiro retirado foi \textbf{3}\newline
$\frac{1}{5} \times \frac{4}{4} = \frac{4}{20}$ Para o caso de que o primeiro retirado foi \textbf{4}\newline

Observe que o caso que o primeiro retirado é 5 não é válido porque não tem numero maior que esse\newline

Assim, vamos somar todas essas frações e teremos $\frac{10}{20} = 0.5$
\end{solution}

\begin{problem}{4}
\textbf{[0,037pts]} Questao\_4 Suponha que uma moeda é jogada três vezes. Se assumimos que as jogadas são independentes e a probabilidade de caras é p, encontre a probabilidade dos eventos nenhuma coroa, uma coroa, duas coroas e três coroas. 

\end{problem}

\begin{solution}

Para resolver essa questao basta verificar as probabilidades:

p - probabilidade de sair CARA
k - probabilidade de sair COROA

$p+K=1$ assim, $k = 1-p$

Dessa forma,

(a) Probabilidade de não sair coroa:

P(i) = $p\times p \times p = p^3$

(b) Probabilidade de sair uma coroa:

P(b) = $C(3,1) \times p \times p \times (1-p) = 3 p^2 (1-p)$ 

(c) Probabilidade de sair duas coroa:

P(c) = $C(3,2) \times p \times (1-p) \times (1-p) = 3 p (1-p)^2$ 

(d) Probabilidade de sair 3 coroas:

P(d) = $(1-p) \times (1-p) \times (1-p) = (1-p)^3$ 

\end{solution}

\begin{problem}{5}
\textbf{[0,037pts]} Questao\_5 Uma companhia tem três máquinas B1, B2 e B3 que fabricam resistores de $1k\Omega$ Observou-se  que $80\%$ dos resistores produzidos por B1 têm tolerância de $50\Omega$do valor nominal. A máquina B2 produz 90\% dos resistores com tolerância de $50\Omega$ do valor nominal. A porcentagem para a 
máquina  B3  é  de  60\%.  A  cada  hora,  a  máquina  B1  produz  3000  resistores,  B2  produz  4000 
resistores,  e  B3  produz 3000  resistores.  Todos os resistores  são misturados  em  um  recipiente 
comum e  empacotados  para  envio.  Desenhe  um  diagrama  em  árvore  para este  experimento. 
Qual a probabilidade de escolher um resistor da máquina B2 com tolerância maior que 50?

\end{problem}

\begin{solution}

Dados da questão:

\noindent $B_{1}$ = 80\% resistores com tolerancia de 50$\Omega \newline$
$B_{2}$ = 90\% resistores com tolerancia de 50$\Omega \newline$
$B_{3}$ = 90\% resistores com tolerancia de 50$\Omega \newline$

\noindent $B_{1}$ = A cada hora produz 3000 resistores \newline
$B_{2}$ = A cada hora produz 4000 resistores \newline
$B_{3}$ = A cada hora produz 3000 resistores \newline

Dessa, forma teremos:

$\frac{80}{100} \times 3000 = 2400$ \newline

$\frac{90}{100} \times 4000 = 3600$ \newline

$\frac{60}{100} \times 3000 = 1800$ \newline

Observe o que esses valores nos dizem, em $B_{1}$ temos 3000 resistores, desses 3000 2400 resistores tem tolerancia e 600 não tem. Assim:

Fazendo:

\noindent $B_{j}$ = O resistor da máquina $B_{j}$ \newline
T = O resistor de tolerância de 50$\Omega$

Assim, 

$P(B_{j} \cap T)$ = probabilidade do resistor ser da caixa $B_{j}$ e ter tolerância de 50$\Omega$ 

$P(T|B_{2})$ = probabilidade do resistor ter tolerancia maior que 50 $\Omega$ dado que pertence a caixa $B_{2}$

Assim: $P(B_{2} \cap T) = P(B_{2} \times P(T|B_{2})$

Temos que: 

$P(T|B_{2})= 0,1$ 

$P(B_{2} \cap T) = \frac{4000}{3000+4000+3000} = 0,4$ \newline

Por isso: $P(B_{2} \cap T) = 0,4 \times 0,1 = 0,04$
\end{solution}


\begin{problem}{6}
\textbf{[0,037pts]} Questao\_6 Considere o jogo do Três. Você embaralha um baralho de três cartas: às, 2 e 3. Se o às vale um ponto, você retira cartas do baralho até que a soma seja 3 ou mais. Você ganha se o total for 3. Calcule P[W], a probabilidade de vencer o jogo.

\end{problem}

\begin{solution}

Observe que o espaço amostral é o conjunto dos pares ordenados das retiradas de cartas:

$S = {(1,2), (1,3), (2,1), (2,3) (3)}$

Em qual desses casos vencemos ?

$A = {(1,2), (2,1), (3)}$

Assim, $P[W] = \frac{3}{5}$

\end{solution}

\begin{problem}{7}
\textbf{[0,037pts]} Questao\_7 Quatro moedas ideais são arremessadas simultaneamente.

(a) Quantos resultados são possíveis?
(b) Associe probabilidades adequadas para a obtenção de quatro coroas, uma cara, duas caras, 
três caras e quatro caras neste experimento.

\end{problem}

\begin{solution}

a) 

C - Evento CARA
K - Evento COROA

Espaço amostral é tal que:

$S={(0000), (0001), (0010), (0011), (0100), (0101), (0110), (0111), (1000), (1001), (1010), (1011), (1100), (1101), (1110), (1111)}$

Assim a quantidade de resultados possíveis são 16

b) 
(i) A possibilidade de 4 coroas:

P(i) = $\frac{1}{16}$

(ii) A probabilidade de 1 cara

P(ii) = $\frac{4}{16} = \frac{1}{4}$

(iii) A probabilidade de 2 caras

P(iii) = $\frac{6}{16} = \frac{3}{8}$

(iv)  A probabilidade de 3 caras

P(iv) = $\frac{4}{16} = \frac{1}{4}$

(v)   A probabilidade de 4 caras

P(v) = $\frac{1}{16}$

\end{solution}

\begin{problem}{8}
\textbf{[0,037pts]} Questao\_8 Três dados não viciados são jogados. Calcule as probabilidades dos eventos de se obter uma 
soma de 8, 9 e 10 pontos. 

\end{problem}

\begin{solution}

Antes de resolvermos a questão é preciso definir que o dado se trata de um dado normal com 6 lados (1-6)

Em vista disso, o espaço amostral é $6 \times 6 \times 6 = 216$ possibilidades 

(i) Evento soma obtida igual a 8

Para isso ser verdade temos os valores: (1,1,6) (1,4,3) (1,5,2) (2,3,3) (2,2,4)

Observe que para os eventos (1,1,6) (2,3,3) e (2,2,4) o número de possibilidades é equivalente ao número de permutações distintas igual a uma combinação de 2 pra 3

$C= \frac{3!}{2! \times (3 - 2)!} = 3$

Para os eventos (1,4,3) e (1,5,2) temos um arranjo de 3 pra 3

$A=3 \times 2 \times 1 = 6$

Por isso a probabilidade da soma ser 8 é:

$P(i) = \frac{3 \times 2 + 3 \times 6 + 1}{216} = \frac{25}{216}$

(ii) Evento soma obtida igual a 10

Para isso ser verdade temos os valores: (1,3,6) (1,5,4) (2,2,6) (2,3,5) (4,3,3) (4,4,2)

Novamente aplicando combinação e arranjo temos:

Combinação de 2 pra 3 para (2,2,6), (4,3,3) (4,4,2)

$C= \frac{3!}{2! \times (3 - 2)!} = 3$

Arranjo 3 pra 3 para (1,3,6) (1,5,4) (2,3,5)

$A=3 \times 2 \times 1 = 6$

Por isso a probabilidade da soma ser 10 é:

$P(ii) = 3 \frac{\times 3 + 6 \times 3}{216} = \frac{27}{216}$

\end{solution}

\begin{problem}{9}
\textbf{[0,037pts]} Questao\_9 Uma certa cidade tem 8 faróis aleatoriamente localizados, quatro dos quais ficam verdes por 
meio minuto na direção leste-oeste e meio minuto na direção nortesul, três permanecem verdes 
por  1/4  de  minuto  na  direção  leste-oeste  e  3/4  de  minuto  na  direção  norte-sul,  e  o  último 
permanece verde 3/4 de minuto na direção leste-oeste e 1/4 de minuto na direção norte-sul. 
Assuma que todos os faróis são independentes, isto é, não existe nenhum tipo de sincronização 
entre eles.

Um automóvel está viajando de forma aleatória através da cidade. Encontre: 

a) a probabilidade de o automóvel encontrar um sinal verde na direção leste-oeste. 

b) a probabilidade de o automóvel encontrar um sinal verde na direção norte-sul. 

c)Qual  é  a  probabilidade  de  um  automóvel viajando  aleatoriamente  pela  cidade encontre  um 
sinal verde?

\end{problem}

\begin{solution}

TOTAL = 8 faróis

Tipo A - 4 primeiros faróis
Tipo B - 3 faróis seguintes
Tipo C - último farol

A probabilidade de pertencer aos faróis do Tipo A

Eventos:

(i) Probabilidade do veículo viajar entre os faróis do Tipo A

(ii) Probabilidade do veículo viajar entre os faróis do Tipo B

(iii) Probabilidade do veículo viajar entre os faróis do Tipo C

(iv) Probabilidade de viajar na direção leste-oeste

(v) Probabilidade de viajar na direção leste-oeste E trafegando em um dos faróis do Tipo A 

(vi) Probabilidade de viajar na direção leste-oeste E trafegando em um dos faróis do Tipo B

(vii) Probabilidade de viajar na direção leste-oeste E trafegando em um dos faróis do Tipo C

(viii) Probabilidade de viajar na direção norte-sul

Assim,

$P(iv) = P(i) \cap P(iv) + P(ii) \cap P(iv) + P(iii) \cap P(iv) $
Dessa forma,

$ P(iv) = \frac{4}{8} \times \frac{1}{2} + \frac{3}{8} \times \frac{1}{4} + \frac{1}{8} \times \frac{3}{4} = \frac{14}{32} = \frac{7}{16}$

b) 

Parecido com o que foi feito antes podemos fazer exatamente a mesma coisa substituindo P(iv) por P(viii)

Assim,

$P(viii) = \frac{4}{8} \times \frac{1}{2} + \frac{3}{8} \times \frac{3}{4} + \frac{1}{8} \times \frac{1}{4} = \frac{18}{32} = \frac{9}{16}$

Observe que isso satisfaze o teorema da probabilidade: $P(iv) + P(viii) = 1$

c)
Uma vez que o tempo de um sinal estar verde ou vermelho é exatamente o mesmo e que na questão não é dito nenhuma condição de alternância entre eles podemos supor o caso ideal onde a probabilidade seria de $50\%$ para verde assim como para vermelho

\end{solution}

\begin{problem}{10}
\textbf{[0,037pts]} Questao\_10 Uma urna contém 3 bolas vermelhas e 2 brancas. Duas bolas são retiradas em sucessão, a 
primeira bola sendo recolocada antes da retirada da segunda. 

(a) Quantos resultados são possíveis? 

(b) Associe probabilidades a cada um destes resultados.

\end{problem}

\begin{solution}

a) $S = {(V,V), (B,B), (V,B), (B,V)}$

b)

$P(V,V) = \frac{3}{5} \times \frac{3}{5} = \frac{9}{25}$

$P(B,B) = \frac{2}{5} \times \frac{2}{5} = \frac{4}{25}$

$P(V,B) = \frac{3}{5} \times \frac{2}{5} = \frac{6}{25}$

$P(B,V) = \frac{2}{5} \times \frac{3}{5} = \frac{6}{25}$

\end{solution}

\begin{problem}{11}
\textbf{[0,037pts]} Questao\_11 Repita o problema anterior se a primeira bola não for recolocada antes da segunda retirada.


\end{problem}

\begin{solution}

a) O espaço amostral não muda: $S = {(V,V), (B,B), (V,B), (B,V)}$

b)

$P(V,V) = \frac{3}{5} \times \frac{2}{4} = \frac{6}{20}$

$P(B,B) = \frac{2}{5} \times \frac{1}{4} = \frac{2}{20}$

$P(V,B) = \frac{3}{5} \times \frac{2}{4} = \frac{6}{20}$

$P(B,V) = \frac{2}{5} \times \frac{3}{4} = \frac{6}{20}$


\end{solution}

\begin{problem}{12}
\textbf{[0,037pts]} Questao\_12 No problema anterior, se sabemos que a primeira retirada foi de uma bola branca, qual é a probabilidade de a segunda retirada ser também de uma bola branca ?
\end{problem}

\begin{solution}

$P[B|B] = \frac{P[B \cap B]}{P[B]} = \frac{\frac{2}{20}}{\frac{2}{5}} = \frac{5}{20} = \frac{1}{4}$

\end{solution}

\begin{problem}{13}
\textbf{[0,037pts]} Questao\_13 No  problema  11),  se  sabemos  que  a  segunda  bola  é  vermelha,  qual  a  probabilidade  de  a primeira também ter sido vermelha? Qual a probabilidade da primeira bola ter sido branca?
\end{problem}

\begin{solution}

Eventos:

(i) A primeira bola ser vermelha

(ii) A segunda bola ser vermelha

P(ii|i) Probabilidade da segunda bola ser vermelha dado que a primeira foi vermelha

P(i|ii) Probabilidade da primeira bola ser vermelha dado que a segunda foi vermelha

$P(i|ii) = \frac{P(ii|i) \times P(i)}{P(ii)}$

Onde, $P(ii) = P(ii \cap i) + P(ii \cap B)$

Assim, $P(ii) = P(i) \times P(ii|i) + P(B) \times P(ii|B)$

$P(ii) = \frac{3}{5} \times \frac{2}{4} + \frac{2}{5} \times \frac{3}{4} = \frac{12}{20} = \frac{3}{5}$

$P(i|ii) = \frac{\frac{2}{4} \times \frac{3}{5}}{\frac{3}{5}} = \frac{2}{4} = \frac{1}{2}$

b)

P(B,V) Probabilidade da primeira bola ter sido branca dado que a segunda foi vermelha

$P(B,ii) = \frac{P(ii|B) \times P(B)}{P(ii)} = \frac{\frac{3}{4} \times \frac{2}{5}}{\frac{3}{5}} = \frac{1}{2}$

\end{solution}

\begin{problem}{14}
\textbf{[0,037pts]} Questao\_14 Uma urna contém 3 bolas vermelhas, 5 bolas brancas e 8 bolas pretas. Outra urna contém 6 bolas vermelhas, 7 bolas brancas e 4 bolas pretas. Uma bola é retirada de cada urna. Encontre a probabilidade de obter duas bolas da mesma cor.
\end{problem}

\begin{solution}

Primeira Urna - 3V 5B 8P
Segunda Urna - 6V 7B 4P

Eventos:

$B|U_{1}$ A bola ser branca dado que pertence a urna 1
$B|U_{2}$ A bola ser branca dado que pertence a urna 1
$V|U_{1}$ A bola ser branca dado que pertence a urna 1
$V|U_{2}$ A bola ser branca dado que pertence a urna 1
$P|U_{1}$ A bola ser branca dado que pertence a urna 1
$P|U_{2}$ A bola ser branca dado que pertence a urna 1
$X$ evento da questão

Então, para resolver é simples, basta encontrar a probabilidade de VV BB ou PP

$P(X) = P(B|U_{1}) \times (B|U_{2}) + (V|U_{1}) \times (V|U_{1}) + (P|U_{1}) \times (P|U_{1})$

$P(X) = \frac{5}{16} \times \frac{7}{17} + \frac{3}{16} \times \frac{6}{17} + \frac{8}{16} \times \frac{4}{17} = \frac{85}{272} $

\end{solution}

\begin{problem}{15}
\textbf{[0,037pts]} Questao\_15 A caixa I contém 3 bolas vermelhas e 5 bolas brancas, e a caixa II, 4 vermelhas e 2 brancas. Extrai-se ao acaso uma bola da primeira caixa e coloca-se na segunda, sem observar a cor. Extrai-se então uma bola da segunda caixa. Qual a probabilidade da mesma ser branca?
\end{problem}

\begin{solution}

Caixa I - $3V_{1} 5B_{1}$
Caixa II - $4V_{2} 2B_{2}$

Evento 1: A bola que sai da caixa 1 para a caixa 2 é branca

Para isso, temos que a bola que sai tanto da primeira caixa quanto da segunda é branca (E)

$P(B_{2}|1) = P(B_{1})P(B_{2}|B_{1} = \frac{5}{8} \times \frac{3}{7} = \frac{15}{56}$

Evemtp 2: A bola que sai da caixa 1 para a caixa 2 é vermelha

Para isso, temos que a bola que sai da primeira caixa é vermelha e da segunda é branca

$P(B_{2}|2) = P(V_{1})P(B_{2}|V_{1} = \frac{3}{8} \times \frac{2}{7} = \frac{6}{56}$

$P(B_{2}) = P(B_{2}|1) + P(B_{2}|2) = \frac{15}{56} + \frac{6}{56} = \frac{21}{56}$

\end{solution}

\begin{problem}{16}
\textbf{[0,037pts]} Questao\_16 Em certo colégio, 25\% dos estudantes foram reprovados em matemática, 15\% em química 
e 10\% em matemática e química ao mesmo tempo. Um estudante é selecionado aleatoriamente.

a)  Se  ele  foi  reprovado  em  química,  qual  é  a  probabilidade  de  ele  ter  sido  reprovado  em 
matemática? 

b) Se  ele  foi  reprovado  em  matemática,  qual  é  a  probabilidade  de  ele  ter  sido  reprovado  em 
química? 

c) Qual é a probabilidade de ele ter sido reprovado em matemática ou química?

\end{problem}

\begin{solution}

M - Matemática
Q - Química

a)

$P(M|Q) = \frac{P(M \cap Q)}{P(Q)} = \frac{\frac{10}{100}}{\frac{15}{100}} = \frac{10}{100} \times \frac{100}{15} = \frac{2}{3}$

b)
$P(Q|M) = \frac{P(Q \cap M)}{P(M)} = \frac{\frac{10}{100}}{\frac{25}{100}} = \frac{10}{100} \times \frac{100}{25} = \frac{2}{5}$

c)
$P(Q \cup M) = P(Q) + P(M) - P(Q \cap M) = \frac{15}{100} + \frac{25}{100} - \frac{10}{100} = \frac{30}{100} = 30\%$


\end{solution}

\begin{problem}{17}
\textbf{[0,037pts]} Questao\_17 
\end{problem}

\begin{solution}

\end{solution}

\begin{problem}{18}
\textbf{[0,037pts]} Questao\_18 Uma  urna  contém  duas  bolas  pretas  e  três  bolas  brancas.  Duas  bolas  são  selecionadas aleatoriamente da urna sem reposição, e a sequência de cores é anotada. Encontre a 
probabilidade de retirar duas bolas pretas.
\end{problem}

\begin{solution}

Urna: 2P 3B

$P(P) = \frac{2}{5} \times \frac{1}{4} = \frac{1}{10}$


\end{solution}

\begin{problem}{19}
\textbf{[0,037pts]} Questao\_19 Lança-se  uma moeda viciada de  modo que  P[cara] =  2/3 e  P[coroa] =  1/3. Se aparecer 
cara,  então  seleciona-se  aleatoriamente  um  número  dentre  os  de  1  a  9;  se  aparecer  coroa, 
seleciona-se aleatoriamente um número dentre os de 1 a 5. Encontre a probabilidade p de um 
número par ser selecionado.
\end{problem}

\begin{solution}

C - {1,2,3,4,5,6,7,8,9}
K - {1,2,3,4,5}

$P{par} = P(par \cap C) + P(par \cap K)$

$P(par) = P(C)P(par|C) + P(K)P(par|K)$

$P(par) = \frac{2}{3} \times \frac{4}{9} + \frac{1}{3} \times \frac{2}{5} = \frac{8}{27} + \frac{2}{15} = \frac{58}{135}$

\end{solution}

\begin{problem}{20}
\textbf{[0,037pts]} Questao\_20 Dois  dígitos  são  selecionados  aleatoriamente  de  1  a  9,  sem  reposição.  Se  a  soma  é  par, encontre a probabilidade p de ambos os números serem ímpares.
\end{problem}

\begin{solution}

S: {1,2,3,4,5,6,7,8,9}

P - Evento par, os dois numeros serem pares
I - Evento impar, os dois números serem impares

Espaco amostral de P: $S_{P} = C(4,2) = \frac{4!}{2!(4 - 2)!} = 6$

Espaco amostral de I: $S_{I} = C(5,2) = \frac{5!}{2!(5 - 2)!} = 10$

$P(I) = \frac{S_{I}}{S_{I} + S_{P}} = \frac{10}{16} = \frac{5}{8}$

\end{solution}

\begin{problem}{21}
\textbf{[0,037pts]} Questao\_21 elefones celulares realizam handoffs à medida em que se movem de uma célula para outra. 
Suponha que durante uma chamada, os telefones realizam zero handoffs (H0), um handoff (H1), ou dois handoffs (H2). Adicionalmente, cada chamada pode ser longa (L) ou breve (B). 

Sabendo que P[L,H0] = 0.1, P[B,H1] = 0.1, P[H2] = 0.3, P[B] = 0.6 e P[H0] =0.5, calcule: 

(a) A probabilidade de não ocorrer nenhum handoff durante uma chamada. 

(b) A probabilidade de uma chamada ser breve.

(c) A probabilidade de uma chamada ser longa ou existirem pelo menos dois handoffs.

\end{problem}

\begin{solution}

Dados:

B - Breve \newline
L - Longa

P(B) = 0,6
P(L) = 0,4

$P(H_{0}) = 0,5$
$P(L \cap H_{0}) = P(H_{0} \cap L) 0,1$
\end{solution}

\begin{problem}{22}
\textbf{[0,037pts]} Questao\_22
\end{problem}

\begin{solution}


\end{solution}

\begin{problem}{23}
\textbf{[0,037pts]} Questao\_23
\end{problem}

\begin{solution}


\end{solution}

\begin{problem}{24}
\textbf{[0,037pts]} Questao\_24
\end{problem}

\begin{solution}


\end{solution}

\begin{problem}{25}
\textbf{[0,037pts]} Questao\_25
\end{problem}

\begin{solution}


\end{solution}

\begin{problem}{26}
\textbf{[0,037pts]} Questao\_26
\end{problem}

\begin{solution}


\end{solution}

\begin{problem}{27}
\textbf{[0,037pts]} Questao\_27
\end{problem}

\begin{solution}


\end{solution}

\begin{problem}{28}
\textbf{[0,037pts]} Questao\_28
\end{problem}

\begin{solution}


\end{solution}

\end{document}




